\documentclass{article}
\usepackage{amsmath}
\begin{document}
Outlining 

\begin{enumerate}
\item What did we do? 
\begin{itemize}
\item Developed the theory for an unbiased, finite variance estimator for the quantity $\frac{\partial E}{\partial p}$.
\item Developed a practical protocol for implementing this cutoff estimator and discussed the bias-variance trade-off inherent to the implementation.
\item Demonstrated the benefits of using the cutoff estimator in optimization schemes for a simple system (CuO molecule?)
\end{itemize} 
\item Who would care about this paper?

\begin{itemize}
\item QMC optimization methods which rely on efficient, low-variance, evaluation of the energy parameter gradient will benefit from the cutoff estimator. 
\item A specific case is the energy optimization scheme of Tolouse, Umrigar and the line minimization scheme that our group is using now.
\end{itemize} 

\item How do we exhibit the importance of the work? 
\begin{itemize}
\item Emphasize the importance of evaluating the parameter gradient $\frac{\partial E}{\partial p}$ accurately and efficiently in QMC.
\begin{itemize}
\item Useful in model fitting 
\item Useful in energy optimization of many-body wave functions
\end{itemize}
\item Demonstrate the issue with using the naive MC estimator for the gradient.
\begin{itemize}
\item Divergences in the wave function gradient and local energy are present near the nodal surface of the wave function $\Psi$.
\item This motivates us to break the integral needed to evaluate the energy gradient into an integral near the nodal surface and one away from it.
\item The divergences present in the prior integral cancel to yield a finite result for  $\frac{\partial E}{\partial p}$ if evaluated exactly.
\item When estimated over a finite set of MC samples, the naive MC estimator for the first integral has an infinite variance.
\end{itemize}
\item Express how the cutoff estimator fixes the issue.
\begin{itemize}
\item The cutoff estimator has an identical expected value when compared to the naive estimator but has a finite variance.
\item In practice, a finite bias is incurred when reducing the variance of the estimator, however the bias can be controlled by the cutoff parameter.
\end{itemize}
\item Demonstrate the benefits of using the cutoff estimator in energy optimization.
\begin{itemize}
\item \textbf{Not super clear how to go about this, need to plan a bit more.}
\end{itemize}
\end{itemize} 
\end{enumerate}

\end{document}