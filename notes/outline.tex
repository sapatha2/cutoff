\documentclass{article}
\usepackage{amsmath}
\begin{document}
\section{Introduction}
%Nice first sentence: Low-variance MC estimates of the expectation value of the Hamiltonian and its parameter derivatives play an integral role in the efficiency of QMC wavefunction optimization techniques.
\begin{enumerate}
\item \textbf{Overall goal } The development of efficient optimization schemes for correlated trial wave functions has been a long standing problem in quantum Monte Carlo (QMC) simulations of realistic systems.

\item \textbf{Barrier to achieving that goal } A significant drain on the efficiency of these schemes is the infinite variance of a naive estimator for the parameter derivative of the expectation value of the energy.

\item \textbf{Current state of the art} The most widely accepted solutions to this infinite variance problem include Sorella's reweighting scheme [ref] and the improved estimators proposed by Assaraf and Caffarel [ref].

\item \textbf{Our advancement to the state of the art } In this work, we derive a simple regularization of the naive estimator which handles the infinite variance problem without relying on auxiliary or guiding wave functions.
\end{enumerate}

\section{Regularized estimator}
\begin{enumerate}
\item The infinite variance of the naive Monte Carlo estimate of $\frac{\partial E}{\partial p}$ is caused by the divergence of the estimator near the nodes of $\Psi$.

\item The variance can be made finite by regularizing the naive estimator within a distance $\epsilon$ from the nodes, while leaving it bare away from this region.

\item An additional constraint on the regularization function allows us to tune the bias of the estimator to be negligible, precisely $O(\epsilon^3)$. 

\item The final regularization function, which satisfies all of these constraints, takes a simple, even polynomial form.

\item This regularized estimator serves as a lightweight, flexible alternative to the commonly used reweighting and improved estimator methods.

\item A practical algorithm for using this regularized estimator follows three steps. 
\end{enumerate}

\section{Application to LiH molecule}

\section{Conclusions}

\pagebreak
\end{document}